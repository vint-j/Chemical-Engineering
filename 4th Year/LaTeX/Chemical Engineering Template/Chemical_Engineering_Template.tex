\documentclass[a4paper, 12pt]{article}
\usepackage[margin=1in]{geometry} % for setting 1 inch margins
\usepackage{graphicx}      % for figures
\usepackage{amsmath}       % for math symbols and equations
\usepackage{chemfig}       % for chemical structures
\usepackage{siunitx}       % for SI units
\usepackage{float}         % for positioning figures/tables
\usepackage{booktabs}      % for better tables
\usepackage{hyperref}      % for hyperlinks
\usepackage{lipsum}        % for dummy text
\usepackage{tabularx}      % for better tables (often used in academic writing)

\title{Chemical Engineering Report Title}
\author{Your Name}
\date{\today}

\begin{document}

\maketitle

\begin{abstract}
This report discusses the analysis and results of [describe experiment or project]. The focus is on [key aspects of the work] and the application of chemical engineering principles to solve [problem or challenge].
\end{abstract}

\section{Introduction}
Chemical engineering is an interdisciplinary field involving the application of physics, chemistry, mathematics, and biology. In this report, we will investigate [brief overview of the topic or process], focusing on [key process variables, reaction kinetics, etc.].

\lipsum[1-2]  % Adding some dummy text for bulk

\section{Theory and Equations}
This section introduces the theoretical framework for the analysis. Important equations include mass balances, energy balances, and rate equations. For example, a general mass balance for a system is given by:

\[
\frac{dM}{dt} = \text{Inflow} - \text{Outflow} + \text{Generation} - \text{Consumption}
\]

For reaction kinetics, we use the following rate equation:

\[
r = k \cdot C_A^n
\]

where:
\begin{itemize}
    \item $r$ = reaction rate
    \item $k$ = reaction rate constant
    \item $C_A$ = concentration of species A
    \item $n$ = reaction order
\end{itemize}

\lipsum[3]  % Adding some dummy text for bulk

\section{Experimental Setup}
Describe the experimental setup, equipment, and materials used in the study. Use diagrams and figures where necessary. For example:

\begin{figure}[H]
    \centering
    \includegraphics[width=0.5\textwidth]{example-image.png}
    \caption{Schematic of the experimental setup.}
    \label{fig:setup}
\end{figure}

\lipsum[4]  % Adding some dummy text for bulk

\section{Results and Discussion}
Present the results of the experiment or simulation. Discuss the significance of the results in terms of chemical engineering principles. Include tables and graphs where necessary.

\begin{table}[H]
    \centering
    \begin{tabularx}{\linewidth}{Xccc}
    \toprule
    Description & Temperature (\si{\celsius}) & Conversion (\%) & Reaction Rate (\si{\mole\per\liter\per\second}) \\
    \midrule
    Sample 1 & 100 & 90 & 0.02 \\
    Sample 2 & 150 & 95 & 0.03 \\
    Sample 3 & 200 & 98 & 0.04 \\
    \bottomrule
    \end{tabularx}
    \caption{Reaction rate at different temperatures.}
    \label{tab:results}
\end{table}

\section{Chemical Structure Example}
Below is an example of a chemical structure drawn using the \texttt{chemfig} package:

\begin{center}
    \chemfig{H-C(-[2]H)(-[6]H)-C(-[2]H)(-[6]H)-OH}
\end{center}

This represents an ethanol molecule.

\section{Conclusion}
Summarize the key findings and their relevance to chemical engineering applications. Discuss possible improvements or future work related to the project.

\lipsum[5]  % Adding some dummy text for bulk

\section{References}
Use a consistent citation format. Example:

\begin{thebibliography}{9}
\bibitem{ref1} Author Name, \emph{Title of the Book}, Edition, Publisher, Year.
\bibitem{ref2} Author Name, Title of the Article, \emph{Journal Name}, Volume, Page Numbers, Year.
\end{thebibliography}

\end{document}